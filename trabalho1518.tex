\documentclass[12pt]{article}%parametros iniciais como fonte 12pt
\usepackage{amsmath}%controle sobre alinhamentos e criacao de equacoes
\usepackage{graphicx}%possibilidade de incluir imagens \includegraphics{image.png}
\usepackage{hyperref}
\usepackage{pgfplots}
\pgfplotsset{compat=1.18}
\usepackage{color}
\usepackage{pgf}
\usepackage[utf8]{inputenc}
\usepackage[brazil]{babel}
\title{Tareda de Casa }
\author{Angelo Gabriel Soares dos Santos}
\date{\today}
\begin{document}
\maketitle
\section{Pagina 15(Exercícios1 e  2)}
\textbf{Exercícios:}

1. O quadro abaixo apresenta as estaturas, em centímetros, de uma turma de 48 alunos do BIOPARK. Construir a tabela de distribuição de frequências em classes.

\begin{center}
\begin{tabular}{|c|c|c|c|c|c|c|c|}
\hline
182 & 164 & 180 & 186 & 167 & 171 & 161 & 180 \\ \hline
176 & 174 & 165 & 180 & 185 & 172 & 165 & 174 \\ \hline
179 & 163 & 177 & 170 & 186 & 186 & 186 & 171 \\ \hline
160 & 166 & 187 & 162 & 163 & 166 & 167 & 187 \\ \hline
174 & 169 & 175 & 161 & 182 & 183 & 160 & 186 \\ \hline
174 & 176 & 188 & 169 & 173 & 160 & 161 & 183  \\ \hline
\end{tabular}
\end{center}
\color[rgb]{0.8,0.2,0.2}

\textbf{Resoluçao:}
\color[rgb]{0.2,0.2,0.2}

Tabela ordenada em ordem crescente:
\begin{center}
\begin{tabular}{|c|c|c|c|c|c|c|c|}
\hline
160 & 160 & 160 & 161 & 161 & 161 & 162 & 163 \\ \hline
163 & 163 & 164 & 165 & 165 & 166 & 166 & 167 \\ \hline
167 & 167 & 169 & 169 & 170 & 171 & 171 & 171 \\ \hline
172 & 173 & 174 & 174 & 174 & 175 & 176 & 176 \\ \hline
177 & 179 & 180 & 180 & 180 & 182 & 182 & 183 \\ \hline
183 & 185 & 186 & 186 & 186 & 186 & 187 & 188 \\ \hline
\end{tabular}
\end{center}


Tabela com classes, frequência, frequência acumulada, frequência relativa e frequência relativa acumulada.
\begin{center}
\begin{tabular}{c|c|c|c|c}
\hline
    Classes & $f_i$ & $F_i$ & $fr_i$ &  $Fr_i$\\\hline
       160  &     3   &    3   &     \pgfmathparse{100*3/48}\pgfmathresult\%   &    \pgfmathparse{100*3/48}\pgfmathresult\%    \\\hline
    161   &     3  &   6    &    \pgfmathparse{100*3/48}\pgfmathresult\%    &     \pgfmathparse{100*6/48}\pgfmathresult\%   \\\hline
    162   &     1   &     7  &    \pgfmathparse{100*1/48}\pgfmathresult\%    &    \pgfmathparse{100*7/48}\pgfmathresult\%    \\\hline
     163  &     3   &    10   &     \pgfmathparse{100*3/48}\pgfmathresult\%   &    \pgfmathparse{100*10/48}\pgfmathresult\%    \\\hline
    164    &    1    &    11   &    \pgfmathparse{100*1/48}\pgfmathresult\%    &    \pgfmathparse{100*11/48}\pgfmathresult\%    \\\hline
     165    &   2     &  13     &    \pgfmathparse{100*2/48}\pgfmathresult\%    &   \pgfmathparse{100*13/48}\pgfmathresult\%     \\\hline
     166    &     2   &  15    &   \pgfmathparse{100*2/48}\pgfmathresult\%     &    \pgfmathparse{100*15/48}\pgfmathresult\%    \\\hline
    167     &     3   &    18   &    \pgfmathparse{100*3/48}\pgfmathresult\%    &   \pgfmathparse{100*18/48}\pgfmathresult\%     \\\hline
      169   &    2    &    20   &   \pgfmathparse{100*2/48}\pgfmathresult\%     &    \pgfmathparse{100*20/48}\pgfmathresult\%    \\\hline
     170    &    1    &   21    &    \pgfmathparse{100*1/48}\pgfmathresult\%    &    \pgfmathparse{100*21/48}\pgfmathresult\%    \\\hline
       171      &     3   &    24   &    \pgfmathparse{100*3/48}\pgfmathresult\%    &    \pgfmathparse{100*24/48}\pgfmathresult\%   \\\hline
     172    &      1  &   25    &  \pgfmathparse{100*1/48}\pgfmathresult\%      &    \pgfmathparse{100*25/48}\pgfmathresult\%    \\\hline
    173     &    1    &   26    &    \pgfmathparse{100*1/48}\pgfmathresult\%    &    \pgfmathparse{100*26/48}\pgfmathresult\%    \\\hline
    174     &     3   &   29   &   \pgfmathparse{100*3/48}\pgfmathresult\%     &    \pgfmathparse{100*29/48}\pgfmathresult\%    \\\hline
    175     &     1   &    30   &   \pgfmathparse{100*1/48}\pgfmathresult\%     &   \pgfmathparse{100*30/48}\pgfmathresult\%     \\\hline
    176     &      2  &   32    &   \pgfmathparse{100*2/48}\pgfmathresult\%     &   \pgfmathparse{100*32/48}\pgfmathresult\%     \\\hline
    177     &     1   &   33    &    \pgfmathparse{100*1/48}\pgfmathresult\%    &   \pgfmathparse{100*33/48}\pgfmathresult\%     \\\hline
    179     &    1    &    34   &    \pgfmathparse{100*1/48}\pgfmathresult\%    &   \pgfmathparse{100*34/48}\pgfmathresult\%     \\\hline
    180    &    3    &    37   &   \pgfmathparse{100*3/48}\pgfmathresult\%     &    \pgfmathparse{100*37/48}\pgfmathresult\%    \\\hline
    182     &    2    &   39    &   \pgfmathparse{100*2/48}\pgfmathresult\%     &   \pgfmathparse{100*39/48}\pgfmathresult\%     \\\hline
    183    &      2  &    41   &    \pgfmathparse{100*2/48}\pgfmathresult\%    &    \pgfmathparse{100*41/48}\pgfmathresult\%    \\\hline
    185     &    1    &    42   &   \pgfmathparse{100*1/48}\pgfmathresult\%     &   \pgfmathparse{100*42/48}\pgfmathresult\%     \\\hline
    186     &     4   &   46    &  \pgfmathparse{100*4/48}\pgfmathresult\%      &    \pgfmathparse{100*46/48}\pgfmathresult\%    \\\hline
    187     &     1   &   47    &   \pgfmathparse{100*1/48}\pgfmathresult\%     &   \pgfmathparse{100*47/48}\pgfmathresult\%     \\\hline
    188    &    1    &    48   &    \pgfmathparse{100*1/48}\pgfmathresult\%    &    \pgfmathparse{100*48/48}\pgfmathresult\%    \\\hline
    Total &       &       &        &        \\\hline
    
\end{tabular}
\end{center}
2.Os dados abaixo representam as medidas de uma dimensão de uma peca(em mm)
 produzida por um processode usinagem. Construir a tabelade distribuição de frequencias em classes.
 
 102,8-136,4-110,1-115,9-118,5-149,3-125,3-144,8-129,7-132,7
 135,0-108,2-138,1-138,6-139,6-144,4-125,9-145,2-145,7-120,4
 \break Colocando as medidas em uma tabela:
\begin{center}
    \begin{tabular}{|c|c|c|c|c|}
    \hline
        102,08 & 136,4 & 110,1 & 115,9 & 118,5 \\\hline
        149,3 & 125,3 & 144,8 & 129,7 & 132,7 \\\hline
        135,0 & 108,2 & 138,1 & 138,6 & 139,6 \\\hline
        144,4 & 125,9 & 145,2 & 145,7 & 120,4 \\\hline 
    \end{tabular}
\end{center}
Ordenando elementos:
\begin{center}
    \begin{tabular}{|c|c|c|c|c|}
    \hline
        102,08 & 108,2 & 110,1 & 115,9 & 118,5 \\\hline
        120,4 & 125,3 & 125,9 & 129,7 & 132,7 \\\hline
        135,0 & 136,4 & 138,1 & 138,6 & 139,6 \\\hline
        144,4 & 144,8 & 145,2 & 145,7 & 149,3 \\\hline 
    \end{tabular}
\end{center}
 Determinando numero de classes($k$):
 
 $k=$ \sqrt{n}
 
 $k=$ \sqrt{20}

 $k=$ 4,472135954999579


Determinando $AT$:
 
$AT=$ 149,3 - 102,08

$AT=$ \pgfmathparse{149.3-102.08}\pgfmathresult

Determinando Amplitude da classe($A_{i}$):
$A_{i}=$ \[\frac{47.22}{4.472135954999579}\]
\newcommand{\AC}{10.55871298975401}
$A_{i}=$ 10,55871298975401

\begin{center}
\begin{tabular}{c|c|c|c|c}
\hline
    Classes & $f_i$ & $F_i$ & $fr_i$ &  $Fr_i$\\\hline
    102,08-\pgfmathparse{102.08+\AC}\pgfmathresult & 3 & 3 & \pgfmathparse{(3/20)*100}\pgfmathresult\%& \pgfmathparse{(3/20)*100}\pgfmathresult\%\\\hline
    \pgfmathparse{102.08+(\AC)}\pgfmathresult-\pgfmathparse{102.08+(\AC*2)}\pgfmathresult & 3 & 6 & \pgfmathparse{(3/20)*100}\pgfmathresult\% & \pgfmathparse{(6/20)*100}\pgfmathresult\% \\\hline
    \pgfmathparse{102.08+(\AC*2)}\pgfmathresult-\pgfmathparse{102.08+(\AC*3)}\pgfmathresult & 4 & 10 & \pgfmathparse{(4/20)*100}\pgfmathresult\% & \pgfmathparse{(10/20)*100}\pgfmathresult\%\\\hline
    \pgfmathparse{102.08+(\AC*3)}\pgfmathresult-\pgfmathparse{102.08+(\AC*4)}\pgfmathresult & 5 & 15 &  \pgfmathparse{(5/20)*100}\pgfmathresult\% & \pgfmathparse{(15/20)*100}\pgfmathresult\%\\\hline
    \pgfmathparse{102.08+(\AC*4)}\pgfmathresult-149,3 & 5 & 20 & \pgfmathparse{(5/20)*100}\pgfmathresult\% & \pgfmathparse{(20/20)*100}\pgfmathresult\%\\\hline

\end{tabular}
\end{center}
\section{Pagina 18(Exercícios2 e  3)}
\textbf{Exercícios:}
2. Os transdutores de temperatura de um determinado tipo são enviados em lotes de
 50.Uma amostra de 60 lotes foi selecionada e o número de transdutores fora das
 especificações em cada lote foi determinado, resultando nos dados a seguir:\\
 2-1-2-4-0-1-3-2-0-5-3-3-1-3-2-4-7-0-2-3
 0-4-2-1-3-1-1-3-4-1-2-3-2-2-8-4-5-1-3-1
 5-0-2-3-2-1-0-6-4-2-1-6-0-3-3-3-6-1-2-3
 \\ Colocando dados em uma tabela: 

 \begin{center}
        \begin{tabular}{|c|c|c|c|c|c|c|c|c|c|}
        \hline
        2 & 1 & 2 & 4 & 0 & 1 & 3 & 2 & 0 & 5 \\
        \hline
        3 & 3 & 1 & 3 & 2 & 4 & 7 & 0 & 2 & 3 \\
        \hline
        0 & 4 & 2 & 1 & 3 & 1 & 1 & 3 & 4 & 1 \\
        \hline
        2 & 3 & 2 & 2 & 8 & 4 & 5 & 1 & 3 & 1 \\
        \hline
        5 & 0 & 2 & 3 & 2 & 1 & 0 & 6 & 4 & 2 \\
        \hline
        1 & 6 & 0 & 3 & 3 & 3 & 6 & 1 & 2 & 3 \\
        \hline
        \end{tabular}
\end{center}
Colocando dados em orden crescente
\begin{center}
        \begin{tabular}{|c|c|c|c|c|c|c|c|c|c|}
        \hline
   0 & 0 & 0 & 0 & 0 & 0 & 0 & 1 & 1 & 1 \\\hline   
   1 & 1 & 1 & 1 & 1 & 1 & 1 & 1 & 1 & 2  \\\hline
   2 & 2 & 2 & 2 & 2 & 2 & 2 & 2 & 2 & 2 \\\hline
   2 & 2 & 3 & 3 & 3 & 3 & 3 & 3 & 3 & 3\\\hline
   3 & 3 & 3 & 3 & 3 & 3 & 4 & 4 & 4 & 4\\\hline
   4 & 4 & 5 & 5 & 5 & 6 & 6 & 6 & 7 & 8\\ \hline
        \end{tabular}
\end{center}
 a)Determine as frequencias e frequencias relativas dos valores observados de x=
 número de transdutores fora das especificações em um lote.
 \color[rgb]{0.6,0.2,0.2}
 \begin{center}
     \begin{tabular}{c|c|c}
         Classes & $f_{i}$ & $fr_{i}$ \\\hline
          0    &  7 &  \pgfmathparse{(7/60)*100}\pgfmathresult\% \\\hline
          1    &  12 &  \pgfmathparse{(12/60)*100}\pgfmathresult\% \\\hline
          2    &  13 &  \pgfmathparse{(13/60)*100}\pgfmathresult\% \\\hline
          3    &  14 &  \pgfmathparse{(14/60)*100}\pgfmathresult\% \\\hline
          4    &  6 &  \pgfmathparse{(6/60)*100}\pgfmathresult\% \\\hline
          5    &  3 &  \pgfmathparse{(3/60)*100}\pgfmathresult\% \\\hline
          6    &  3 &  \pgfmathparse{(3/60)*100}\pgfmathresult\% \\\hline
          7    &  1 &  \pgfmathparse{(1/60)*100}\pgfmathresult\% \\\hline
          8    &  1 &  \pgfmathparse{(1/60)*100}\pgfmathresult\% \\\hline
          
     \end{tabular}
 \end{center}
 \color[rgb]{0.2,0.2,0.2}
  b)Que proporção de lotes na amostra possui no maximo cinco transdutores fora
 das especificações? Que proporção tem menos de cinco? Que proporção possui
 no minimo cinco unidades fora das especificações?\\
 \color[rgb]{0.6,0.2,0.2}
 
 \pgfmathparse{((7+12+13+14+6+3)/60)*100}\pgfmathresult\% dos lotes tem no maximo cinco transdutores fora das especificacoes, \\\pgfmathparse{((7+12+13+14+6)/60)*100}\pgfmathresult\% tem menos de cinco transdutores fora das especificacoes, \\\pgfmathparse{((3+3+1+1)/60)*100}\pgfmathresult\% possui no minimo cinco transdutores fora das especificacoes, 
 \color[rgb]{0.2,0.2,0.2}
 
\\\\ c)Desenhe um histograma dos dados, usando a frequencia relativa na escala vertical e comente suas caracteristicas.

\begin{center}
\begin{tikzpicture}
    \begin{axis}[
        width=10cm,
        height=6cm,
        xlabel={Transdutores fora das especificaçoes},
        ylabel={Frequencia de anomalias},
        grid=major,
        ybar=2*\pgflinewidth,
        bar width=15pt,
        xticklabel style={rotate=45},
        xtick={1,2,3,4,5,6,7,8,9},
        xticklabels={
        0,
        1,
        2,
        3,
        4,
       5,
        6,
       7,
        8
        },
        ymin=0
        ]
        \addplot[
            ybar,
            bar width=15pt,
            fill=blue
        ] 
        coordinates {
        (1,11.66534) 
        (2,19.9997) 
        (3,21.66595) 
        (4,23.33221) 
        (5,9.99908)
        (6,4.99878)
        (7,4.99878)
        (8,1.66626)
        (9,1.66626)

        
        };
    \end{axis}
\end{tikzpicture}
\end{center}
OBS: no grafico notamos que a linha vertical mostra a frequencia relativa de transdutores fora das especificacoes assim como requisitado no exercicio logo na linha horisontal deduzimos que dos 60 lotes 7 nao apresentaram anomalias 12 apenas uma, 13 2 e assim por diante.
\\
 3.Onumero de particulas de contaminação de uma pastilha de silício antes de certo
 processo de limpeza foi determinado para cada pastilha em uma amostra de tamanho
 100, resultando nas frequencias a seguir:
 \begin{center}
     \begin{tabular}{c|c|c|c|c|c|c|c|c|c|c|c|c|c|c|c}
     \hline
         N° de Particulas & 0 & 1 & 2 & 3&4&5&6&7&8&9&10&11&12&13&14 \\\hline
         Frequencia & 1&2&3&12&11&15&18&10&12&4&5&3&1&2&1\\\hline
     \end{tabular}
 \end{center}
\end{document}
