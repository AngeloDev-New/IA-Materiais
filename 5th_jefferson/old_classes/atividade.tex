
\documentclass[a4paper,12pt]{article}
\usepackage[utf8]{inputenc}
\usepackage[brazil]{babel}
\usepackage{enumitem}
\usepackage{geometry}
\geometry{margin=2.5cm}

\begin{document}

\section{perguntas a responder}

\begin{enumerate}[leftmargin=*, label=\arabic*.]
    \item Qual é o produto mais vendido em cada estado?

    \item A última promoção de biscoitos foi realmente mais eficaz no sul do que no nordeste?

    \item Como podemos nos preparar para a alta demanda de protetor solar no verão, sabendo o que aconteceu nos anos anteriores?

    \item \textbf{Dados Essenciais:} Quais tipos de dados a EcoMarket precisa coletar e centralizar para responder às perguntas do diretor? Considere informações como vendas, clientes, produtos e tempo.
    
    \vspace{3cm} % Espaço para resposta

    \item \textbf{Por que centralizar?} Por que não é prático simplesmente acessar o sistema de cada loja (o chamado sistema OLTP) individualmente para responder a essas perguntas? O que aconteceria se tentássemos fazer isso?
    
    \vspace{3cm}

    \item \textbf{Benefícios Estratégicos:} Quais seriam os benefícios concretos que a EcoMarket teria ao implementar um Data Warehouse? Considere aspectos como facilidade de análise, velocidade e qualidade das decisões.
    
    \vspace{3cm}

    \item \textbf{Análises Possíveis:} Cite três exemplos de análises que seriam possíveis com os dados centralizados. Pense em algo que não é viável hoje e que traria grande valor para o negócio.
    
    \vspace{3cm}

    \item \textbf{Fluxo de Dados:} Em um papel ou quadro, desenhe uma estrutura simplificada que mostre o fluxo de dados. Comece nas lojas, passe por um processo de limpeza e transformação e termine no repositório central (Data Warehouse). Não precisa ser técnico! Use caixas, setas e textos simples para ilustrar a ideia.
    
    \vspace{4cm}
\end{enumerate}

\end{document}
